%%%%%%%%%%%%%%%%%%%%%%%%%%%%%%%%%%%%%%%%%
% Beamer Presentation
% LaTeX Template
% Version 1.0 (10/11/12)
%
% This template has been downloaded from:
% http://www.LaTeXTemplates.com
%
% License:
% CC BY-NC-SA 3.0 (http://creativecommons.org/licenses/by-nc-sa/3.0/)
%
%%%%%%%%%%%%%%%%%%%%%%%%%%%%%%%%%%%%%%%%%

%----------------------------------------------------------------------------------------
%	PACKAGES AND THEMES
%----------------------------------------------------------------------------------------

\documentclass{beamer}

\mode<presentation> {

% The Beamer class comes with a number of default slide themes
% which change the colors and layouts of slides. Below this is a list
% of all the themes, uncomment each in turn to see what they look like.

%\usetheme{default}
%\usetheme{AnnArbor}
%\usetheme{Antibes}
%\usetheme{Bergen}
%\usetheme{Berkeley}
%\usetheme{Berlin}
%\usetheme{Boadilla}
%\usetheme{CambridgeUS}
%\usetheme{Copenhagen}
%\usetheme{Darmstadt}
%\usetheme{Dresden}
%\usetheme{Frankfurt}
%\usetheme{Goettingen}
%\usetheme{Hannover}
%\usetheme{Ilmenau}
%\usetheme{JuanLesPins}
%\usetheme{Luebeck}
\usetheme{Madrid}
%\usetheme{Malmoe}
%\usetheme{Marburg}
%\usetheme{Montpellier}
%\usetheme{PaloAlto}
%\usetheme{Pittsburgh}
%\usetheme{Rochester}
%\usetheme{Singapore}
%\usetheme{Szeged}
%\usetheme{Warsaw}

% As well as themes, the Beamer class has a number of color themes
% for any slide theme. Uncomment each of these in turn to see how it
% changes the colors of your current slide theme.

%\usecolortheme{albatross}
%\usecolortheme{beaver}
%\usecolortheme{beetle}
%\usecolortheme{crane}
%\usecolortheme{dolphin}
%\usecolortheme{dove}
%\usecolortheme{fly}
%\usecolortheme{lily}
%\usecolortheme{orchid}
%\usecolortheme{rose}
%\usecolortheme{seagull}
%\usecolortheme{seahorse}
%\usecolortheme{whale}
%\usecolortheme{wolverine}

%\setbeamertemplate{footline} % To remove the footer line in all slides uncomment this line
%\setbeamertemplate{footline}[page number] % To replace the footer line in all slides with a simple slide count uncomment this line

%\setbeamertemplate{navigation symbols}{} % To remove the navigation symbols from the bottom of all slides uncomment this line
}

\usepackage{graphicx} % Allows including images


\begin{document}

\section{Multi-Output Prediction Model}

\begin{frame}{Bayesian additive regression tree (BART)}
\begin{itemize}
\item Nonparametric prediction tool
\item Sum of T 'weak learner' trees % that leads to robust out-of-sample predictions
\item Dynamically learns important predictors via a sparsity inducing prior
%\item Excels when $x$ is high in dimension, with relatively few important predictors
\end{itemize}

Mathematically, a BART model looks like:

\begin{align*}
Y_i &\sim N(\mu_i, \sigma^2)\\
\mu_i &= \sum_{t=1}^T g(x_i ; \tau_t, M_t)%\sum_{l \in L_t}\psi_{lt}I(x_i \leadsto (t,l))
\end{align*}



%$I(x_i \leadsto (t,l)) = 1$ if $x_i$ falls into node ($t,l$) of tree $t$; 0 otherwise. 

\end{frame}

\begin{frame}{Shared Forest}
\begin{itemize}
\item An extension of BART for multivariate responses
\item Correlation between responses is utilized via shared tree structures
\item Variables that predict $Y_1$ are likely the same variables that predict $Y_2$ (though the nature of the relationship may be different!)
\item Information sharing $\Rightarrow$ better predictions for $Y_1$ \textit{and} $Y_2$ 
\end{itemize}
\end{frame}



\section{Simulation Study} 


\subsection{Binary/Continuous} 

\begin{frame}{1 binary, 1 continuous }
\begin{align}
Y_i &\sim N(\mu_i, \sigma^2_i) \\
\delta_i &\sim Bernoulli(\pi_i)
\end{align}

\begin{itemize}
\item $\begin{pmatrix}\mu_i \\ \pi_i \end{pmatrix}$ modeled jointly using a shared forest model

$$\begin{pmatrix}\mu_i \\ \Phi^{-1}(\pi_i) \end{pmatrix} = 
\begin{pmatrix}\sum_{t=1}^T \sum_{l \in L_t}\psi_{lt}I(x_i \leadsto (t,l)) \\ \sum_{t=1}^T \sum_{l \in L_t}\theta_{lt}I(x_i \leadsto (t,l)) \end{pmatrix} $$
\item Likelihood involves two dimensional response: $\{Y_i, \delta_i\}_{i = 1}^N$
\end{itemize}
\end{frame}


\begin{frame}{Simulation Study }{Setup}
We run 100 simulations. In each:
\begin{itemize}
\item $n_{train} = 500$, $n_{test} = 500$
\item We set the number of covariates to P = 150. 
\item $x_1, \hdots, x_{150} \sim Unif(0,1)$
\item True underlying means based on a modification of the ``Friedman function"
\begin{align*}
y_i &= 10 \sin (\pi x_{1i} x_{2i}) + 20(x_{3i} - 0.5)^2 + 10x_{4i} + 5x_{5i} +
\epsilon_i^Y\\
\delta_i &= \begin{cases}1 &\text{ if }5 \sin (\pi x_{1i} x_{2i}) + 25(x_{3i} - 0.5)^2 + 5x_{4i} + 10x_{5i} +
\epsilon_i^{\delta} > 0 \\
                          0 & \text{ otherwise}\end{cases}
\end{align*}

where $(\epsilon_i^Y, \epsilon_i^{\delta}) \overset{iid}{\sim} N(0,1)$

\end{itemize}



\end{frame}

\begin{frame}{1 binary, 1 continuous }
Two quantities may be of interest for prediction
\begin{enumerate}
\item $E(\delta^*, y^* \mid x)$, where $^*$ indicates those observations come from a test / hold-out sample
  \begin{itemize}
    \item[ex)] Predict salary of individual, conditional on cellular traits $x_i$
    \item[ex)] Predict gender of individual, conditional on cellular traits $x_i$
  \end{itemize}
\item $E(y^* \mid \delta)$
  \begin{itemize}
    \item[ex)] Predict mean salary of females (avg. across traits observed for females)
  \end{itemize}
\end{enumerate}
\end{frame}


\begin{frame}{Predicting Individual $Y^*$ }
\includegraphics[width = .9\linewidth]{continuous_sim_results_ind_y.pdf}
\end{frame}

\begin{frame}{Predicting Individual $\delta*$}
\includegraphics[width = .9\linewidth]{continuous_sim_results_ind_delta.pdf}
\end{frame}


\begin{frame}{Predicting $E(Y^* \mid \delta^*)$ }
\includegraphics[width = .9\linewidth]{continuous_sim_results.pdf}
\end{frame}

\begin{frame}{Predicting $E(Y^* \mid \delta^*)$ }

% latex table generated in R 4.0.5 by xtable 1.8-4 package
% Tue Mar  1 10:17:05 2022
\begin{table}[ht]
\centering
\begin{tabular}{rlrrr}
  \hline
$\delta$ & Model & MSE & L\_bound & U\_bound \\ 
  \hline
0 & BART & 0.004875 & 0.000004 & 0.023207 \\ 
  0 & Shared Forest & 0.004330 & 0.000007 & 0.017697 \\ 
  1 & BART & 0.028464 & 0.000042 & 0.119264 \\ 
  1 & Shared Forest & 0.010399 & 0.000011 & 0.041591 \\ 
   \hline
\end{tabular}
\caption{\label{tab:mse} Mean of the squared errors (across simulations) comparing the true $E(Y^* \mid \delta^*)$ to the estimated $\hat{E}(Y^* \mid \hat{\delta})$.}
\end{table}

\end{frame}

\subsection{Binary/Binary} 

\begin{frame}{2 binary responses}
The model structure is the same as described before; however, the likelihood reflects that:

\begin{align}
\delta_{1i} &\sim Bernoulli(\pi_{1i}) \\
\delta_{2i} &\sim Bernoulli(\pi_{2i})
\end{align}

As before, the tree structures will be built using information shared between $\delta_1$ and $\delta_2$, while the location parameters are estimated separately. 

\end{frame}


\begin{frame}{2 binary responses}
\begin{itemize}
\item In this context, we are interested in predicting $P(\delta^*_2=1 \mid \delta^*_1)$.
\begin{itemize}
\item[ex)] Probability a randomly selected female is employed. \\($\delta_1$ = gender, $\delta_2$ = employment)
\end{itemize}
\item We measure model performance by looking at the squared errors comparing the true population (conditional) means to the estimated. 

\item The simulation study is identical, but with two binary responses. Importantly, these resonses again have differing mean functions. 
\end{itemize}
\end{frame}

\begin{frame}{Predicting $E(\delta_2^* \mid \delta_1^*)$ }
\includegraphics[width = .9\linewidth]{binary_sim_results.pdf}
\end{frame}

\end{document}